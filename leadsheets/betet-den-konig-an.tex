% <title>
% 122 bpm
% https://www.youtube.com/watch?v=pCCpcH4J2Aw

\stepcounter{songnumber}

\begin{tabular}{p{0.6cm}p{12cm}p{1.4cm}}
	\rowcolor{cyan} \myRow{\thesongnumber} & \myRow{Betet den König an} & \myRow{120} \\
	                                       &                            &             \\
\end{tabular}

Verse sind jeweils 5 Zeilen lang, unintuitiv länger

\begin{tabular}{p{1.6cm}l}
	% \textbf{Intro}  & Kick 1, Snare 2, 2a, 3e, 4                          \\
	% \textbf{Vers 1} & wie Intro                                           \\
	% \textbf{Chorus} & Kick 1, 3, 3+, Backbeat                             \\ % original
	%                 & alternativ: 4f, Backbeat                            \\ % mal testen
	% \textbf{Vers 2} & wie Intro                                           \\
	%                 & alternativ: keine HH, nur mit Ghost Notes auf Snare \\
	% \textbf{Chorus} & wie vorher, aber zusätzliche Snare auf 2a           \\
	% \textbf{Bridge} & wie Ch, danach direkt in Ch                         \\
	% \textbf{Chorus} & wie vorher                                          \\
	\textbf{Intro}  & Gustav mäßig                                        \\
	\textbf{Vers 1} & HH schließen, Rim click                             \\
	\textbf{Chorus} & 4f, Backbeat                                        \\ % mal testen
	\textbf{Vers 2} & wie Intro                                           \\
	                & alternativ: keine HH, nur mit Ghost Notes auf Snare \\
	\textbf{Chorus} & wie vorher, aber zusätzliche Snare auf 2a           \\
	\textbf{Bridge} & wie Ch, danach direkt in Ch                         \\
	\textbf{Chorus} & ruhig, Ende steigern                                \\
	\textbf{Chorus} & wie vorher                                          \\
	% \textbf{Ende}   & letzte Zeile insgesamt 3x                           \\
	\textbf{Ende}   & letzte Zeile insgesamt 2x                           \\
	                & 2. mal Half Time                                    \\
	                &                                                     \\
	                & direkt in (3)                                       \\
	                &                                                     \\
\end{tabular}
