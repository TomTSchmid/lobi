% Christ Our Hope In Life And Death
% 76 bpm
% https://www.youtube.com/watch?v=FvwlwL1FUEg

$\frac{3}{4}$ Takt (mit $\frac{2}{4}$ Takten im Chorus eingeschoben)

Übergang Vers in Ch:
\begin{itemize}
	\item Mit letztem Wort starten zwei Takte (je nach Vers steigern)
	\item Dann Akzente auf Viertel (2x Crash, 1x Toms), dzw. bissle auffüllen
	\item Crash auf 1 $\rightarrow$ ganz normal spielen
	      % \item Bzw: 4 Akzente: 2x Crash, 1x Tom, 1x Crash: Immer viertel beginnend mit \textit{Sing}
	      % \item Ende vom Ch kommen zweimal 2/4 Takte: quasi umgedreht spielen
	      % \item Beim zweiten Mal viertel kurz leicht steigern in Intro/Zw
\end{itemize}

\begin{tabular}{p{1.6cm}l}
	\textbf{Start}  & Einzählen, fett                                                         \\
	\textbf{Intro}  & auf die 1 fett reinkommen                                               \\
	\textbf{Vers1}  & Kick auf 1, wenig Toms dazu                                             \\
	\textbf{Zw}     & Ab Ende Vers 4f                                                         \\
	\textbf{Chorus} & erster Takt Ch: Akzent auf jede Viertel (2x Crash, 1x Toms),            \\
	                & dann Crash auf 1, bissle dezent noch halten                             \\
	\textbf{Zw}     & auf die 1 fett reinkommen                                               \\
	\textbf{Vers2}  & Beat mit beiden Händen auf Snare                                        \\
	                & ab Hälfte dann auf HH                                                   \\
	                & direkt in Chorus ohne 2 Takte steigern                                  \\
	\textbf{Chorus} & erster Takt Ch: eher straight halten, nicht voll auf die Crash          \\
	\textbf{Zw}     & auf die 1 fett reinkommen, vor der letzten hohen Note schon weg         \\
	\textbf{Vers3}  & Anfang ruhig, ab Hälfte Kick auf 1, Toms dazu, ab ¾ mehr, Ende steigern \\
	\textbf{Chorus} & steigern bis \textit{sing} (2 Schläge), \textit{Halle} ist              \\
	                & A capella, auf \textit{lu} kommt man wieder rein                        \\
	\textbf{Chorus} & nochmal, Ende einmal wiederholen                                        \\
	                & Letzte Zeile nochmal wiederholen, dann erst Outro                       \\
	\textbf{Outro}  & wie Intro                                                               \\
\end{tabular}
