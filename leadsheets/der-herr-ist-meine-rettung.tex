% The Lord is my Salvation / Der Herr ist meine Rettung
% 69 bpm
% https://www.youtube.com/watch?v=MzZaeKr3PD4

\stepcounter{songnumber}

\begin{tabular}{p{0.6cm}p{12cm}p{1.4cm}}
	\rowcolor{cyan} \myRow{\thesongnumber} & \myRow{Der Herr ist meine Rettung} & \myRow{69} \\
	                                       &                                    &            \\
\end{tabular}

\begin{tabular}{p{1.6cm}l}
	                & Am Ende vom Chorus kommen 3 Akzente: mitspielen                      \\
	                & letzte Zeile Chorus immer schon bissle ruhiger                       \\
	                &                                                                      \\
	\textbf{Intro}  & nur Klavier                                                          \\
	\textbf{Vers 1} & nur Klavier, ruhig                                                   \\
	\textbf{Vers 2} & Kick auf 1, bissle HH                                                \\
	\textbf{Zw}     & erst ruhig halten, dann zwei Schläge auf Toms                        \\
	\textbf{Chorus} & HH, Kick 1, 2+, 3; Rimclick auf 4, kein Backbeat                     \\
	                & Akzente am Ende mit Kick mitspielen                                  \\
	\textbf{Zw}     & komplett weg, dann offene HH und kurzer Fill als Einleitung          \\
	\textbf{Vers 3} & Gustav mit Rimclick und kleinen Variationen auf Kick                 \\
	                & Kurzer Fill                                                          \\
	\textbf{Vers 4} & Gustav mit Snare und kleinen Kickvariationen                         \\
	\textbf{Zw}     & mit Fill-In: einmal Ansetzen aber nur halb,                          \\
	                & dann nochmal ansetzen und durchziehen                                \\
	\textbf{Chorus} & wie vorher aber Snare richtig, nach Textende runterkommen            \\
	\textbf{Vers 5} & erster Teil komplett ruhig, Akzente auf „Call me home“: Kick (sanft) \\
	                & dann in letzte Zeile reinkommen, dann fett steigern                  \\
	\textbf{Chorus} & wie vorher: satt                                                     \\
	\textbf{Bridge} & auf Ride, Kick 1, 1+, 3, 3+; Backbeat                                \\
	                & 2x                                                                   \\
	\textbf{Ende}   & letzte Zeile 3x, drittes mal auf „der Herr“ raus                     \\
	% \textbf{Ende}   & Fett, weg auf letztes „Salvation“                                    \\
	% \textbf{Chorus} & fett (HH und Ride gleichzeitig)                                      \\
	% \textbf{Instr}  & fett                                                                 \\
	% \textbf{Instr}  & ruhig                                                                \\
	% \textbf{Chorus} & ruhig                                                                \\
	% \textbf{Ende}   & ruhig                                                                \\
\end{tabular}
