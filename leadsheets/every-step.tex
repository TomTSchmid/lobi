% Every Step
% 104 bpm
% https://www.youtube.com/watch?v=-Q-jT_x-pCU

\stepcounter{songnumber}

\begin{tabular}{p{0.6cm}p{12cm}p{1.4cm}}
	\rowcolor{cyan} \myRow{\thesongnumber} & \myRow{Every Step} & \myRow{104} \\
	                                       &                    &             \\
\end{tabular}

\begin{tabular}{p{1.6cm}l}
	\textbf{Intro}  & 4f, Snare auf 2, 3e, 4, voll auf Ride                                    \\
	\textbf{Vers 1} & Kick auf 1, dann über zwei Toms                                          \\
	                & Letze Zeile doppelt so schnell                                           \\
	\textbf{Zw}     & 4f, über Toms mit Snare auf 4                                            \\
	\textbf{Vers 2} & wie Vers 1                                                               \\
	\textbf{Zw}     & 4f, Toms dazu, keine Snare                                               \\
	\textbf{Chorus} & Toms auf 2, 4                                                            \\
	                & Ende A capella (weg auf „Face to face“)                                  \\
	\textbf{Zw}     & wie Intro                                                                \\
	\textbf{Vers 3} & 4f, über Toms, Snare auf 4                                               \\
	                & Ende wie die anderen Verse (Pattern doppelt so schnell)                  \\
	\textbf{Chorus} & wie Vers 3, bissle fetter                                                \\
	                & Ende hochsteigern, dann direkt ruhig werden für Bridge                   \\
	\textbf{Bridge} & erstes Mal nur bissle Ride                                               \\
	                & Einmal fett auf Standtom, 4f                                             \\
	                & Einmal viertel steigern mit Snare                                        \\
	                & Nach Textende 2 Takte steigern, dann Break auf die 4                     \\
	\textbf{Chorus} & Anfang Break, dann ganz normal weiter auf 1                              \\
	                & Beat wie Intro                                                           \\
	% \textbf{Outro}  & letzte Zeile 3x (immer fett), dann fettes Outro (nur 1x) \\
	% \textbf{Outro}  & letzte Zeile 1x fett wiederholen, dann fettes Outro (nur 1x) \\
	\textbf{Outro}  & letzte Zeile 1x fett wiederholen                                         \\
	                & auf \textit{face} liegen lassen, Takt zu Ende Break, dann fett ins Outro \\
	                & Outro nur 1x                                                             \\
	                & Ende ausklingen lassen und nicht hart fett                               \\
	                &                                                                          \\
\end{tabular}
