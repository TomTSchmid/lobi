% Gotteslamm
% 70 bpm
% https://www.youtube.com/watch?v=XznYlGxWgvI

\stepcounter{songnumber}

\begin{tabular}{p{0.6cm}p{12cm}p{1.4cm}}
	\rowcolor{cyan} \myRow{\thesongnumber} & \myRow{Gotteslamm} & \myRow{70} \\
	                                       &                    &            \\
\end{tabular}

\begin{tabular}{p{1.6cm}l}
	                % & \color{red} noch Tempo von (4)! Bridge ruhig anfangen, dann ausklingen lassen \\
	% \textbf{Intro}  & ruhig, ohne Drums                                             \\
	\textbf{Vers 1} & ruhig, ohne Drums                                                             \\
	\textbf{Zw}     & Bissle Ride                                                                   \\
	\textbf{Vers 2} & ruhig Standtom 1, 3, 3a, Sammy-Click auf 4                                    \\
	\textbf{Chorus} & ruhig: Kick 1, bissle Ride dazu                                               \\
	                & jeden zweiten Takt Sammy-Click auf 4                                          \\
	\textbf{Vers 3} & Kick 1, (3) Snare 2, 2a, Rest Ghost Notes (\sechzehntel), ohne HH             \\
	\textbf{Chorus} & 4f, übers Set (erst nach 2), Snare auf 4, noch dezent halten                  \\
	                & Ende kurz \achtel hochsteigern                                                \\
	\textbf{Bridge} & Anfang kurz sacken lassen,                                                    \\ % alternativ
	                & dann 4f und immer wieder Snarewirbel mit vielen Akzenten                      \\ % alternativ
	                & ab Hälfte in vollen satten Beat spielen                                       \\ % alternativ
	\textbf{Chorus} & einmal fett, einmal ruhig                                                     \\ % alternativ
	% \textbf{Bridge} & 4f, durchgehender Snarewirbel mit vielen Akzenten             \\ % original
	%                 & immer wieder kurzer Fill übers Set                            \\ % original
	% \textbf{Chorus} & entspannter aber voller Beat auf Ride                         \\ % original
	% \textbf{Chorus} & einmal ruhig                                                  \\ % original
	                &                                                                               \\
\end{tabular}
