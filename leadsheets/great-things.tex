% Great Things
% 102 bpm
% https://www.youtube.com/watch?v=y4CY3nf1Mvw

\stepcounter{songnumber}

\begin{tabular}{p{0.6cm}p{12cm}p{1.4cm}}
	\rowcolor{cyan} \myRow{\thesongnumber} & \myRow{Great Things} & \myRow{102} \\
	                                       &                      &             \\
\end{tabular}

\begin{tabular}{p{1.6cm}l}
	% \textbf{Intro}  & 1x Thema nix                                                   \\
	\textbf{Intro}  & 2x Thema Disco                                                 \\
	\textbf{Vers 1} & nix                                                            \\
	                & Kick 1, 1+, nächster Takt dumpfe Snare auf 4                   \\
	% \chorus         & bissle Beat (Kick 1, 2, 3; Snare 4)                            \\
	\chorus         & Kick 2, 4                                                      \\
	                & bissle Beat (Kick 1, 2, 3; Snare 4)                            \\
	% & Gustav                                                         \\
	                & Ende (\textit{great things}) nix                               \\
	\textbf{TurnA}  & Disco wie Intro                                                \\
	\textbf{Vers 2} & Kick 1, 1+, Backbeat                                           \\
	\chorus         & Disco mit Snarefills dazwischen                                \\
	\textbf{Zw}     & Beat fortsetzen, Ende \textit{bissle} runterbringen für Bridge \\
	\bridge         & 1. Mal: Über Toms (1a, 2+, 3+), 4f dazu, 4 auf Snare           \\
	                & 2. Mal: deutlich mehr Snare und Crashes dazu                   \\
	                & letztes Viertel: \viertel hochsteigern                         \\
	\chorus         & ohne Drums, in Mitte hochsteigern                              \\
	                & für zweite Hälfte dazu kommen: Disco                           \\
	\textbf{Ende}   & Letzte Zeile 2x wiederholen                                    \\
	% \textbf{Outro}  & Thema 2x fett, 1x ganz ruhig                          \\
	\textbf{Outro}  & kurze Pause, dann erst Outro                                   \\
	                &                                                                \\
\end{tabular}
