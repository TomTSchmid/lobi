% https://www.youtube.com/watch?

\stepcounter{songnumber}

\begin{tabular}{p{0.6cm}p{12cm}p{1.4cm}}
    \rowcolor{cyan} \myRow{\thesongnumber} & \myRow{I believe} & \myRow{133} \\
                                           &                   &             \\
\end{tabular}

\begin{tabular}{p{1.6cm}l}
    \textbf{Intro}  & 8 Takte Snare, Kick X, bissle Kick variieren, Ende Snare auf 3, 3+, 4, 4+          \\
    \textbf{Vers 1} & 4f                                                                                 \\
    \textbf{Zw}     & 4 Takte, fett Toms auf 1, 1a, 2+                                                   \\
    \textbf{Vers 2} & 4f mit bissle HH                                                                   \\
    \textbf{Zw}     & 2 Takte hochsteigern                                                               \\
    \textbf{Chorus} & nix, Ende hochsteigern (\textit{I believe})                                        \\
    \textbf{Zw}     & 8 Takte, wie Intro                                                                 \\
    \textbf{Vers 3} & 4f mit bissle HH                                                                   \\
    \textbf{Vers 4} & 4f, Backbeat dazu                                                                  \\
    \textbf{Zw}     & 2 Takte (teilweise mit Text), nur Akzente auf 1, 1a, 2+, 3e, 4                     \\
                    & Ende hochsteigern                                                                  \\
    \textbf{Chorus} & 4f, satt Toms dazu, \achtel steigern auf \textit{I}, weg sein auf \textit{believe} \\
    \textbf{Zw}     & 8 Takte, direkt mit Textende ruhig                                                 \\
    \textbf{Bridge} & nix, ruhig                                                                         \\
                    & 4f, HH \sechzehntel dazu                                                           \\
                    & bissle steigern                                                                    \\
    \textbf{Chorus} & Fett Kick und Standtom auf 1, 1a, 2+                                               \\
                    & ab \textit{The King who was and is} wie vorher im Chorus                           \\
    \textbf{Zw}     & 2 Takte, wie Intro                                                                 \\
    \textbf{Chorus} & fett, 4f, Snare 1a, 2+, 3a, 4+                                                     \\
    \textbf{Ende}   & ohne Outro, direkter Übergang in (4)                                               \\
                    &                                                                                    \\
\end{tabular}
